% Options for packages loaded elsewhere
\PassOptionsToPackage{unicode}{hyperref}
\PassOptionsToPackage{hyphens}{url}
%
\documentclass[
]{book}
\usepackage{lmodern}
\usepackage{amssymb,amsmath}
\usepackage{ifxetex,ifluatex}
\ifnum 0\ifxetex 1\fi\ifluatex 1\fi=0 % if pdftex
  \usepackage[T1]{fontenc}
  \usepackage[utf8]{inputenc}
  \usepackage{textcomp} % provide euro and other symbols
\else % if luatex or xetex
  \usepackage{unicode-math}
  \defaultfontfeatures{Scale=MatchLowercase}
  \defaultfontfeatures[\rmfamily]{Ligatures=TeX,Scale=1}
\fi
% Use upquote if available, for straight quotes in verbatim environments
\IfFileExists{upquote.sty}{\usepackage{upquote}}{}
\IfFileExists{microtype.sty}{% use microtype if available
  \usepackage[]{microtype}
  \UseMicrotypeSet[protrusion]{basicmath} % disable protrusion for tt fonts
}{}
\makeatletter
\@ifundefined{KOMAClassName}{% if non-KOMA class
  \IfFileExists{parskip.sty}{%
    \usepackage{parskip}
  }{% else
    \setlength{\parindent}{0pt}
    \setlength{\parskip}{6pt plus 2pt minus 1pt}}
}{% if KOMA class
  \KOMAoptions{parskip=half}}
\makeatother
\usepackage{xcolor}
\IfFileExists{xurl.sty}{\usepackage{xurl}}{} % add URL line breaks if available
\IfFileExists{bookmark.sty}{\usepackage{bookmark}}{\usepackage{hyperref}}
\hypersetup{
  pdftitle={R Shiny Training Resources},
  pdfauthor={Yihui Xie},
  hidelinks,
  pdfcreator={LaTeX via pandoc}}
\urlstyle{same} % disable monospaced font for URLs
\usepackage{color}
\usepackage{fancyvrb}
\newcommand{\VerbBar}{|}
\newcommand{\VERB}{\Verb[commandchars=\\\{\}]}
\DefineVerbatimEnvironment{Highlighting}{Verbatim}{commandchars=\\\{\}}
% Add ',fontsize=\small' for more characters per line
\usepackage{framed}
\definecolor{shadecolor}{RGB}{248,248,248}
\newenvironment{Shaded}{\begin{snugshade}}{\end{snugshade}}
\newcommand{\AlertTok}[1]{\textcolor[rgb]{0.94,0.16,0.16}{#1}}
\newcommand{\AnnotationTok}[1]{\textcolor[rgb]{0.56,0.35,0.01}{\textbf{\textit{#1}}}}
\newcommand{\AttributeTok}[1]{\textcolor[rgb]{0.77,0.63,0.00}{#1}}
\newcommand{\BaseNTok}[1]{\textcolor[rgb]{0.00,0.00,0.81}{#1}}
\newcommand{\BuiltInTok}[1]{#1}
\newcommand{\CharTok}[1]{\textcolor[rgb]{0.31,0.60,0.02}{#1}}
\newcommand{\CommentTok}[1]{\textcolor[rgb]{0.56,0.35,0.01}{\textit{#1}}}
\newcommand{\CommentVarTok}[1]{\textcolor[rgb]{0.56,0.35,0.01}{\textbf{\textit{#1}}}}
\newcommand{\ConstantTok}[1]{\textcolor[rgb]{0.00,0.00,0.00}{#1}}
\newcommand{\ControlFlowTok}[1]{\textcolor[rgb]{0.13,0.29,0.53}{\textbf{#1}}}
\newcommand{\DataTypeTok}[1]{\textcolor[rgb]{0.13,0.29,0.53}{#1}}
\newcommand{\DecValTok}[1]{\textcolor[rgb]{0.00,0.00,0.81}{#1}}
\newcommand{\DocumentationTok}[1]{\textcolor[rgb]{0.56,0.35,0.01}{\textbf{\textit{#1}}}}
\newcommand{\ErrorTok}[1]{\textcolor[rgb]{0.64,0.00,0.00}{\textbf{#1}}}
\newcommand{\ExtensionTok}[1]{#1}
\newcommand{\FloatTok}[1]{\textcolor[rgb]{0.00,0.00,0.81}{#1}}
\newcommand{\FunctionTok}[1]{\textcolor[rgb]{0.00,0.00,0.00}{#1}}
\newcommand{\ImportTok}[1]{#1}
\newcommand{\InformationTok}[1]{\textcolor[rgb]{0.56,0.35,0.01}{\textbf{\textit{#1}}}}
\newcommand{\KeywordTok}[1]{\textcolor[rgb]{0.13,0.29,0.53}{\textbf{#1}}}
\newcommand{\NormalTok}[1]{#1}
\newcommand{\OperatorTok}[1]{\textcolor[rgb]{0.81,0.36,0.00}{\textbf{#1}}}
\newcommand{\OtherTok}[1]{\textcolor[rgb]{0.56,0.35,0.01}{#1}}
\newcommand{\PreprocessorTok}[1]{\textcolor[rgb]{0.56,0.35,0.01}{\textit{#1}}}
\newcommand{\RegionMarkerTok}[1]{#1}
\newcommand{\SpecialCharTok}[1]{\textcolor[rgb]{0.00,0.00,0.00}{#1}}
\newcommand{\SpecialStringTok}[1]{\textcolor[rgb]{0.31,0.60,0.02}{#1}}
\newcommand{\StringTok}[1]{\textcolor[rgb]{0.31,0.60,0.02}{#1}}
\newcommand{\VariableTok}[1]{\textcolor[rgb]{0.00,0.00,0.00}{#1}}
\newcommand{\VerbatimStringTok}[1]{\textcolor[rgb]{0.31,0.60,0.02}{#1}}
\newcommand{\WarningTok}[1]{\textcolor[rgb]{0.56,0.35,0.01}{\textbf{\textit{#1}}}}
\usepackage{longtable,booktabs}
% Correct order of tables after \paragraph or \subparagraph
\usepackage{etoolbox}
\makeatletter
\patchcmd\longtable{\par}{\if@noskipsec\mbox{}\fi\par}{}{}
\makeatother
% Allow footnotes in longtable head/foot
\IfFileExists{footnotehyper.sty}{\usepackage{footnotehyper}}{\usepackage{footnote}}
\makesavenoteenv{longtable}
\usepackage{graphicx}
\makeatletter
\def\maxwidth{\ifdim\Gin@nat@width>\linewidth\linewidth\else\Gin@nat@width\fi}
\def\maxheight{\ifdim\Gin@nat@height>\textheight\textheight\else\Gin@nat@height\fi}
\makeatother
% Scale images if necessary, so that they will not overflow the page
% margins by default, and it is still possible to overwrite the defaults
% using explicit options in \includegraphics[width, height, ...]{}
\setkeys{Gin}{width=\maxwidth,height=\maxheight,keepaspectratio}
% Set default figure placement to htbp
\makeatletter
\def\fps@figure{htbp}
\makeatother
\setlength{\emergencystretch}{3em} % prevent overfull lines
\providecommand{\tightlist}{%
  \setlength{\itemsep}{0pt}\setlength{\parskip}{0pt}}
\setcounter{secnumdepth}{5}

\title{R Shiny Training Resources}
\author{Yihui Xie}
\date{2020-06-04}

\begin{document}
\maketitle

{
\setcounter{tocdepth}{1}
\tableofcontents
}
\begin{Shaded}
\begin{Highlighting}[]
\KeywordTok{library}\NormalTok{(xfun)}
\end{Highlighting}
\end{Shaded}

\begin{center}\rule{0.5\linewidth}{0.5pt}\end{center}

Author: Jimmy Briggs\\
Date: June 2, 2020

\begin{center}\rule{0.5\linewidth}{0.5pt}\end{center}

The purpose of this guide is to provide an in-depth walkthrough of the
various tasks, installations, and processes involved with setting up your
machine for using R.

\hypertarget{installing-r-rstudio}{%
\section{Installing R \& RStudio}\label{installing-r-rstudio}}

The first step is to install R and RStudio from their respective websites listed below.

\hypertarget{download-r}{%
\section{Download R}\label{download-r}}

Install R from the main CRAN (Comprehensive R Archive Network) website:
\url{https://cran.r-project.org/}.

\begin{itemize}
\item
  On the page, select ``Download R for Windows'' \textgreater{} ``Base'' \textgreater{} ``Download R 4.0.0 for Windows''
\item
  Note as of today the latest R version is 4.0.0
\item
  During installation, ensure that you are installing the 64-bit architecture
  version of R for increased RAM.
\end{itemize}

\hypertarget{additional-notes-on-r}{%
\subsection{Additional Notes on R}\label{additional-notes-on-r}}

\begin{itemize}
\item
  CRAN is composed of a set of mirror servers distributed around the world and
  is used to distribute R and R packages.
\item
  Don't try and pick a mirror that's close to you: instead use the cloud mirror,
  \url{https://cloud.r-project.org}, which automatically figures it out for you.
\item
  A new major version of R comes out once a year, and there are 2-3 minor
  releases each year. It's a good idea to update regularly.
\item
  Upgrading can be a bit of a hassle, especially for major versions, which
  require you to re-install all your packages, but putting it off only makes it
  worse.
\end{itemize}

For more details see the section discussing how to efficiently migrate R
packages between versions in this guide.

\begin{itemize}
\tightlist
\item
  To ease the process up updating your R version it is helpful to use the
  \texttt{installr} packge via \texttt{installr::updateR()} (run this from the native R
  console not RStudio).
\end{itemize}

\hypertarget{download-rstudio}{%
\section{Download RStudio}\label{download-rstudio}}

Install the free Version of RStudio Desktop for Windows from the RStudio Website here: \url{https://rstudio.com/products/rstudio/download/}.

\hypertarget{additional-notes-on-rstudio}{%
\subsection{Additional Notes on RStudio}\label{additional-notes-on-rstudio}}

\begin{itemize}
\item
  RStudio is an integrated development environment, or IDE, for R programming.
\item
  RStudio is updated a couple of times a year. When a new version is available,
  RStudio will let you know.
\item
  It's a good idea to upgrade regularly so you can take advantage of the latest
  and greatest features.
\end{itemize}

\begin{center}\rule{0.5\linewidth}{0.5pt}\end{center}

\hypertarget{configure-rstudio-settings}{%
\section{Configure RStudio Settings}\label{configure-rstudio-settings}}

A range of Project Options and Global Options are available in RStudio from the Tools menu (accessible from the keyboard via Alt+T).

Most of these are self-explanatory but it is worth mentioning a few that can boost
your programming efficiency:

\begin{itemize}
\tightlist
\item
  I highly recommend unticking the default ``Restore .RData'' settings box:
\end{itemize}

\emph{Unticking this default prevents loading previously created R objects. This will make starting R quicker and reduce the chance of getting bugs due to previously created objects. For this reason, I recommend you untick this box.}

Alternatively you can simply run this code:

\begin{Shaded}
\begin{Highlighting}[]
\KeywordTok{require}\NormalTok{(usethis)}
\NormalTok{usethis}\OperatorTok{::}\KeywordTok{use\_blank\_slate}\NormalTok{(}\DataTypeTok{scope =} \StringTok{"user"}\NormalTok{)}
\end{Highlighting}
\end{Shaded}

\emph{See \texttt{code\{?usethis::use\_blank\_slate} for more information.}

\begin{itemize}
\item
  GIT/SVN project settings allow RStudio to provide a graphical
  interface to your version control system.
\item
  R version settings allow RStudio to `point' to different R
  versions/interpreters, which may be faster for some projects.
\item
  Code editing options can make RStudio adapt to your coding style, for example,
  by preventing the autocompletion of braces, which some experienced programmers
  may find annoying. Enabling Vim mode makes RStudio act as a (partial) Vim emulator.
\item
  Diagnostic settings can make RStudio more efficient by adding additional
  diagnostics or by removing diagnostics if they are slowing down your work.
  This may be an issue for people using RStudio to analyze large datasets on older
  low-spec computers.
\item
  Appearance: if you are struggling to see the source code, changing the default
  font size may make you a more efficient programmer by reducing the time overheads
  associated with squinting at the screen. Other options in this area relate more
  to aesthetics. Settings such as font type and background color are also important
  because feeling comfortable in your programming environment can boost productivity.
  Go to Tools \textgreater{} Global Options to modify these.
\end{itemize}

\hypertarget{installing-and-setting-up-additional-software-and-utilities}{%
\section{Installing and Setting Up Additional Software and Utilities}\label{installing-and-setting-up-additional-software-and-utilities}}

To get the most out of R and RStudio it is helpful to install these additional
software resources:

\begin{itemize}
\item
  RTools
\item
  Git
\item
  A Git Client GUI (or just use RStudio)
\item
  Git LFS
\item
  Mercurial
\item
  Tex Distribution

  \begin{itemize}
  \tightlist
  \item
    Tinytex
  \item
    Miketex
  \item
    TexLive
  \end{itemize}
\item
  Java
\item
  Pandoc
\item
  Node.js
\item
  Hugo
\item
  Inno
\end{itemize}

Here's the code to install these additional resources:

\begin{Shaded}
\begin{Highlighting}[]

\ControlFlowTok{if}\NormalTok{ (}\OperatorTok{!}\KeywordTok{require}\NormalTok{(pacman)) }\KeywordTok{install.packages}\NormalTok{(}\StringTok{"pacman"}\NormalTok{)}

\NormalTok{pacman}\OperatorTok{::}\KeywordTok{p\_load}\NormalTok{(devtools, }
\NormalTok{               installr, }
\NormalTok{               tinytex, }
\NormalTok{               rstudioapi, }
\NormalTok{               magrittr,}
\NormalTok{               dplyr,}
\NormalTok{               pkgbuild)}

\CommentTok{\# configure RStudio settings {-}{-}{-}{-}{-}{-}{-}{-}{-}{-}{-}{-}{-}{-}{-}{-}{-}{-}{-}{-}{-}{-}{-}{-}{-}{-}{-}{-}{-}{-}{-}{-}{-}{-}{-}{-}{-}{-}{-}{-}{-}{-}{-}{-}{-}{-}}

\CommentTok{\# disable reloading of workspace between sessions}
\NormalTok{usethis}\OperatorTok{::}\KeywordTok{use\_blank\_slate}\NormalTok{(}\DataTypeTok{scope =} \StringTok{"user"}\NormalTok{)}

\CommentTok{\# review system environment variables:}
\KeywordTok{Sys.getenv}\NormalTok{()}

\CommentTok{\# configure your R library path for R packages}
\KeywordTok{.libPaths}\NormalTok{()}

\CommentTok{\# copy packages to new R{-}version\textquotesingle{}s windows library}
\NormalTok{libdir\_prior \textless{}{-}}\StringTok{ }\KeywordTok{file.path}\NormalTok{(}\StringTok{"\textless{}enter prior win{-}library path here\textgreater{}"}\NormalTok{)}
\NormalTok{libdir\_current \textless{}{-}}\StringTok{ }\KeywordTok{file.path}\NormalTok{(}\StringTok{"\textless{}enter current win{-}library path here\textgreater{}"}\NormalTok{)}
\NormalTok{installr}\OperatorTok{::}\KeywordTok{copy.packages.between.libraries}\NormalTok{(}
  \DataTypeTok{from =}\NormalTok{ libdir\_prior, }\DataTypeTok{to =}\NormalTok{ libdir\_current}
\NormalTok{)}

\CommentTok{\# check}
\KeywordTok{.libPaths}\NormalTok{()[}\DecValTok{1}\NormalTok{] }\OperatorTok{==}\StringTok{ }\NormalTok{libdir }

\CommentTok{\# configure dotfiles .Rprofile \& .Renvrion {-}{-}{-}{-}{-}{-}{-}{-}{-}{-}{-}{-}{-}{-}{-}{-}{-}{-}{-}{-}{-}{-}{-}{-}{-}{-}{-}{-}{-}{-}{-}{-}}

\CommentTok{\# review dotfiles}
\NormalTok{usethis}\OperatorTok{::}\KeywordTok{edit\_r\_environ}\NormalTok{(}\DataTypeTok{scope =} \StringTok{"user"}\NormalTok{) }\CommentTok{\# (RTools Path, github PAT, keys, etc.)}
\NormalTok{usethis}\OperatorTok{::}\KeywordTok{edit\_r\_profile}\NormalTok{(}\DataTypeTok{scope =} \StringTok{"user"}\NormalTok{) }\CommentTok{\# (various options for packages)}

\CommentTok{\# additional software {-}{-}{-}{-}{-}{-}{-}{-}{-}{-}{-}{-}{-}{-}{-}{-}{-}{-}{-}{-}{-}{-}{-}{-}{-}{-}{-}{-}{-}{-}{-}{-}{-}{-}{-}{-}{-}{-}{-}{-}{-}{-}{-}{-}{-}}
\NormalTok{pkgbuild}\OperatorTok{::}\KeywordTok{setup\_rtools}\NormalTok{()}

\CommentTok{\# Rtools}
\NormalTok{installr}\OperatorTok{::}\KeywordTok{install.rtools}\NormalTok{()}
\NormalTok{rstudioapi}\OperatorTok{::}\KeywordTok{restartSession}\NormalTok{()}

\CommentTok{\# git}
\NormalTok{installr}\OperatorTok{::}\KeywordTok{install.git}\NormalTok{()}
\NormalTok{rstudioapi}\OperatorTok{::}\KeywordTok{restartSession}\NormalTok{()}

\CommentTok{\# tinytex}
\NormalTok{tinytex}\OperatorTok{::}\KeywordTok{install\_tinytex}\NormalTok{()}
\NormalTok{rstudioapi}\OperatorTok{::}\KeywordTok{restartSession}\NormalTok{()}
\NormalTok{tinytex}\OperatorTok{::}\KeywordTok{use\_tinytex}\NormalTok{()}

\CommentTok{\# java}
\NormalTok{installr}\OperatorTok{::}\KeywordTok{install.java}\NormalTok{()}

\CommentTok{\# pandoc}
\NormalTok{installr}\OperatorTok{::}\KeywordTok{install.pandoc}\NormalTok{()}

\CommentTok{\# node.js (only if desired)}
\NormalTok{installr}\OperatorTok{::}\KeywordTok{install.nodejs}\NormalTok{()}

\CommentTok{\# github Git Client (only if desired)}
\NormalTok{installr}\OperatorTok{::}\KeywordTok{install.github}\NormalTok{()}

\CommentTok{\# inno (only if desired)}
\NormalTok{installr}\OperatorTok{::}\KeywordTok{install.inno}\NormalTok{()}
\end{Highlighting}
\end{Shaded}

Note that not all of these are required, however, I recommend at a minimum to
install RTools, Git, Pandoc, and a Latex service.

I have provided an R script named R-Setup-Script.R with this guide that uses the
\texttt{installr} package to help ease the process of installing all these extra
resources.

\hypertarget{advanced-configuration}{%
\section{Advanced Configuration}\label{advanced-configuration}}

This section discusses more advanced R related configurations such as:

\begin{itemize}
\tightlist
\item
  Environment Paths
\item
  Detailed System Information
\item
  Dotfiles
\item
  Common pitfalls
\end{itemize}

For more advanced R developers you may want to further configure your development
environement by customizing you R related dotfiles;
specifically, your \emph{.Rprofile} and \emph{.Renviron}.

Here is what my minimal setup includes:

Additionally, you can configure keybinding for RStudio addins from RStudio and
store them within the .R folder located in your \emph{R\_USER} path. To view
this path run \texttt{Sys.getenv("R\_USER")}.

On a windows computer, you may need to adjust where the system chooses to look
for various R related items on your system. For example, the path to your RTools
bin executable, your HOME path, your library path for packages,
and many other windows specific paths. Note the difference between
SYSTEM PATHS and USER PATHS.

To add your system RTOOLS PATH to your .Renviron
(easier than manually configuring within windows system settings) run the code:

\begin{Shaded}
\begin{Highlighting}[]
\KeywordTok{cat}\NormalTok{(}\StringTok{\textquotesingle{}PATH = $\{RTOOLS40\_HOME\}}\CharTok{\textbackslash{}\textbackslash{}}\StringTok{usr}\CharTok{\textbackslash{}\textbackslash{}}\StringTok{bin;$\{PATH\}\textquotesingle{}}\NormalTok{,}
    \DataTypeTok{file =}\NormalTok{ fs}\OperatorTok{::}\KeywordTok{path}\NormalTok{(}\KeywordTok{Sys.getenv}\NormalTok{(}\StringTok{"R\_USER"}\NormalTok{), }\StringTok{"/.Renviron"}\NormalTok{), }
    \DataTypeTok{append =} \OtherTok{TRUE}\NormalTok{)}
\end{Highlighting}
\end{Shaded}

YOu can also view the allocated memory RAM your machine allows R to use by
running \texttt{forgot\ the\ function..}.

\hypertarget{getting-started-with-r-and-shiny}{%
\chapter{Getting Started with R and Shiny}\label{getting-started-with-r-and-shiny}}

Below is a collection of resources for getting started in R and Shiny.

They are listed roughly in the order that I would expect a motivated beginner
to work through them.

\hypertarget{getting-stated-with-r}{%
\subsection{Getting Stated with R}\label{getting-stated-with-r}}

\begin{itemize}
\tightlist
\item
  Download R here: \url{https://cran.r-project.org/}
\item
  Download the free version of Rstudio here: \url{https://www.rstudio.com/products/rstudio/download/}
\item
  Getting Started with R: \url{https://rladiessydney.org/courses/ryouwithme/}
\item
  R4DS: \url{https://r4ds.had.co.nz/}

  \begin{itemize}
  \tightlist
  \item
    note: I recommend skimming through the above book and trying out examples.
    Skip over sections that you do not understand, and you can come back to them
    later.
  \end{itemize}
\end{itemize}

\emph{For a more in-depth walkthrough of setting up R see the \href{https://jimbrig.github.io/rtraining/articles/setting-up-r.html}{R Setup Guide} vignette}

\hypertarget{getting-started-with-github}{%
\subsection{Getting Started with GitHub}\label{getting-started-with-github}}

\begin{itemize}
\tightlist
\item
  Create a GitHub account: \url{https://github.com/} . You may need to install git first from here: \url{https://git-scm.com/book/en/v2/Getting-Started-Installing-Git}
\item
  If you are not familiar with git, this article gives a nice overview: \url{https://jahya.net/blog/git-vs-github/}
\item
  Try out example 1 from this tutorial: \url{https://github.com/bcgov/bcgov-data-science-resources/wiki/Tutorial:-Intro-to-Git-\&-GitHub-for-the-R-User}
\end{itemize}

\hypertarget{a-deeper-dive-into-r}{%
\subsection{A Deeper Dive into R}\label{a-deeper-dive-into-r}}

\begin{itemize}
\tightlist
\item
  Fundamental chapters in Advanced R

  \begin{itemize}
  \tightlist
  \item
    \url{http://adv-r.had.co.nz/Data-structures.html}
  \item
    \url{http://adv-r.had.co.nz/Subsetting.html}
  \item
    \url{http://adv-r.had.co.nz/Style.html}
  \end{itemize}
\item
  Introduction to the dplyr package for data manipulation: \url{https://cran.r-project.org/web/packages/dplyr/vignettes/dplyr.html}
\end{itemize}

\hypertarget{getting-started-with-shiny}{%
\subsection{Getting Started with Shiny}\label{getting-started-with-shiny}}

What is Shiny from Rstudio:
\textgreater{} Shiny is an open source R package that provides an elegant and powerful web framework for building web applications using R. Shiny helps you turn your analyses into interactive web applications without requiring HTML, CSS, or JavaScript knowledge.

\begin{itemize}
\tightlist
\item
  Into to Shiny book: \url{https://laderast.github.io/gradual_shiny/}
\item
  Another Intro to Shiny book: \url{https://ourcodingclub.github.io/2017/03/07/shiny.html}
\item
  Video on the basics of web development: \url{https://www.youtube.com/watch?v=FXqTHsPaY0A}
\item
  Shiny basics: \url{https://shiny.rstudio.com/articles/basics.html}
\item
  More Shiny basics: \url{https://deanattali.com/blog/building-shiny-apps-tutorial/} (Dean Attali, the author of this tutorial is one of my clients)
\item
  Articles on Shiny: \url{https://shiny.rstudio.com/articles/}
\end{itemize}

\hypertarget{build-your-own-shiny-app-and-put-it-on-github}{%
\subsection{Build your own Shiny app and put it on GitHub!}\label{build-your-own-shiny-app-and-put-it-on-github}}

\hypertarget{full-books-on-shiny}{%
\subsection{Full Books on Shiny}\label{full-books-on-shiny}}

\begin{itemize}
\tightlist
\item
  Mastering Shiny: \url{https://github.com/hadley/mastering-shiny}
\item
  Engineering Production-Grade Shiny Apps: \url{https://thinkr-open.github.io/building-shiny-apps-workflow/index.html}
\end{itemize}

\hypertarget{want-more}{%
\subsection{Want More?}\label{want-more}}

Here's some more resources gathered from a colleagues recent RShiny learning path:

\begin{itemize}
\item
  basic overview of R in two hour video: \url{https://www.youtube.com/watch?v=_V8eKsto3Ug\&list=PLWKjhJtqVAblQe2CCWqV4Zy3LY01Z8aF1\&index=5\&t=6791s}
\item
  basic overview of shiny web apps(I found it particularly helpful to follow along with the ``\url{https://shiny.rstudio.com/images/shiny-cheatsheet.pdf}''): \url{https://vimeo.com/rstudioinc/review/131218530/212d8a5a7a/\#t=43m32s}
\item
  overview of rstudio IDE and github interactions Part 1 and Part 2 -

  \begin{itemize}
  \tightlist
  \item
    \url{https://resources.rstudio.com/wistia-rstudio-essentials-2/rstudioessentialsmanagingpart1-2}
  \item
    \url{https://resources.rstudio.com/wistia-rstudio-essentials-2/rstudioessentialsmanagingpart2-2}
  \end{itemize}
\item
  useful R for data science chapters:

  \begin{itemize}
  \tightlist
  \item
    workflow:basics link: \url{https://r4ds.had.co.nz/workflow-basics.html}
  \item
    workflow:projects link: \url{https://r4ds.had.co.nz/workflow-projects.html}
  \item
    tibbles link: \url{https://r4ds.had.co.nz/tibbles.html}
  \item
    tidy data link: \url{https://r4ds.had.co.nz/tidy-data.html}
  \item
    functions link: \url{https://r4ds.had.co.nz/functions.html}
  \item
    vectors link:\url{https://r4ds.had.co.nz/vectors.html}
  \item
    iterations with purr link: \url{https://r4ds.had.co.nz/iteration.html}
  \end{itemize}
\item
  learn about dplyr package for data manipulation: \url{https://cran.r-project.org/web/packages/dplyr/vignettes/dplyr.html}
\item
  learn about shiny dashboards: \url{http://rstudio.github.io/shinydashboard/}
\item
  list of cheats sheats and resources found here: \url{https://resources.rstudio.com/}
\item
  highcharter API for rich graphics and charts: \url{http://jkunst.com/highcharter/highcharts-api.html}
\item
  Article on Databases: \url{https://shiny.rstudio.com/articles/persistent-data-storage.html}
\end{itemize}

Built in tutorials for packages can be accessed through the learnr package.
For example if you library learnr and tidyverse, the top right panel of RStudio can display a walk through of different functions in the tidyverse package. Blog post describing package: \url{https://blog.rstudio.com/2020/02/25/rstudio-1-3-integrated-tutorials/}

\end{document}
